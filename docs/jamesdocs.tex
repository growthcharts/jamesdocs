\documentclass[]{book}
\usepackage{lmodern}
\usepackage{amssymb,amsmath}
\usepackage{ifxetex,ifluatex}
\usepackage{fixltx2e} % provides \textsubscript
\ifnum 0\ifxetex 1\fi\ifluatex 1\fi=0 % if pdftex
  \usepackage[T1]{fontenc}
  \usepackage[utf8]{inputenc}
\else % if luatex or xelatex
  \ifxetex
    \usepackage{mathspec}
  \else
    \usepackage{fontspec}
  \fi
  \defaultfontfeatures{Ligatures=TeX,Scale=MatchLowercase}
\fi
% use upquote if available, for straight quotes in verbatim environments
\IfFileExists{upquote.sty}{\usepackage{upquote}}{}
% use microtype if available
\IfFileExists{microtype.sty}{%
\usepackage{microtype}
\UseMicrotypeSet[protrusion]{basicmath} % disable protrusion for tt fonts
}{}
\usepackage{hyperref}
\hypersetup{unicode=true,
            pdftitle={JAMES manual},
            pdfauthor={Stef van Buuren},
            pdfborder={0 0 0},
            breaklinks=true}
\urlstyle{same}  % don't use monospace font for urls
\usepackage{natbib}
\bibliographystyle{apalike}
\usepackage{color}
\usepackage{fancyvrb}
\newcommand{\VerbBar}{|}
\newcommand{\VERB}{\Verb[commandchars=\\\{\}]}
\DefineVerbatimEnvironment{Highlighting}{Verbatim}{commandchars=\\\{\}}
% Add ',fontsize=\small' for more characters per line
\usepackage{framed}
\definecolor{shadecolor}{RGB}{248,248,248}
\newenvironment{Shaded}{\begin{snugshade}}{\end{snugshade}}
\newcommand{\AlertTok}[1]{\textcolor[rgb]{0.94,0.16,0.16}{#1}}
\newcommand{\AnnotationTok}[1]{\textcolor[rgb]{0.56,0.35,0.01}{\textbf{\textit{#1}}}}
\newcommand{\AttributeTok}[1]{\textcolor[rgb]{0.77,0.63,0.00}{#1}}
\newcommand{\BaseNTok}[1]{\textcolor[rgb]{0.00,0.00,0.81}{#1}}
\newcommand{\BuiltInTok}[1]{#1}
\newcommand{\CharTok}[1]{\textcolor[rgb]{0.31,0.60,0.02}{#1}}
\newcommand{\CommentTok}[1]{\textcolor[rgb]{0.56,0.35,0.01}{\textit{#1}}}
\newcommand{\CommentVarTok}[1]{\textcolor[rgb]{0.56,0.35,0.01}{\textbf{\textit{#1}}}}
\newcommand{\ConstantTok}[1]{\textcolor[rgb]{0.00,0.00,0.00}{#1}}
\newcommand{\ControlFlowTok}[1]{\textcolor[rgb]{0.13,0.29,0.53}{\textbf{#1}}}
\newcommand{\DataTypeTok}[1]{\textcolor[rgb]{0.13,0.29,0.53}{#1}}
\newcommand{\DecValTok}[1]{\textcolor[rgb]{0.00,0.00,0.81}{#1}}
\newcommand{\DocumentationTok}[1]{\textcolor[rgb]{0.56,0.35,0.01}{\textbf{\textit{#1}}}}
\newcommand{\ErrorTok}[1]{\textcolor[rgb]{0.64,0.00,0.00}{\textbf{#1}}}
\newcommand{\ExtensionTok}[1]{#1}
\newcommand{\FloatTok}[1]{\textcolor[rgb]{0.00,0.00,0.81}{#1}}
\newcommand{\FunctionTok}[1]{\textcolor[rgb]{0.00,0.00,0.00}{#1}}
\newcommand{\ImportTok}[1]{#1}
\newcommand{\InformationTok}[1]{\textcolor[rgb]{0.56,0.35,0.01}{\textbf{\textit{#1}}}}
\newcommand{\KeywordTok}[1]{\textcolor[rgb]{0.13,0.29,0.53}{\textbf{#1}}}
\newcommand{\NormalTok}[1]{#1}
\newcommand{\OperatorTok}[1]{\textcolor[rgb]{0.81,0.36,0.00}{\textbf{#1}}}
\newcommand{\OtherTok}[1]{\textcolor[rgb]{0.56,0.35,0.01}{#1}}
\newcommand{\PreprocessorTok}[1]{\textcolor[rgb]{0.56,0.35,0.01}{\textit{#1}}}
\newcommand{\RegionMarkerTok}[1]{#1}
\newcommand{\SpecialCharTok}[1]{\textcolor[rgb]{0.00,0.00,0.00}{#1}}
\newcommand{\SpecialStringTok}[1]{\textcolor[rgb]{0.31,0.60,0.02}{#1}}
\newcommand{\StringTok}[1]{\textcolor[rgb]{0.31,0.60,0.02}{#1}}
\newcommand{\VariableTok}[1]{\textcolor[rgb]{0.00,0.00,0.00}{#1}}
\newcommand{\VerbatimStringTok}[1]{\textcolor[rgb]{0.31,0.60,0.02}{#1}}
\newcommand{\WarningTok}[1]{\textcolor[rgb]{0.56,0.35,0.01}{\textbf{\textit{#1}}}}
\usepackage{longtable,booktabs}
\usepackage{graphicx,grffile}
\makeatletter
\def\maxwidth{\ifdim\Gin@nat@width>\linewidth\linewidth\else\Gin@nat@width\fi}
\def\maxheight{\ifdim\Gin@nat@height>\textheight\textheight\else\Gin@nat@height\fi}
\makeatother
% Scale images if necessary, so that they will not overflow the page
% margins by default, and it is still possible to overwrite the defaults
% using explicit options in \includegraphics[width, height, ...]{}
\setkeys{Gin}{width=\maxwidth,height=\maxheight,keepaspectratio}
\IfFileExists{parskip.sty}{%
\usepackage{parskip}
}{% else
\setlength{\parindent}{0pt}
\setlength{\parskip}{6pt plus 2pt minus 1pt}
}
\setlength{\emergencystretch}{3em}  % prevent overfull lines
\providecommand{\tightlist}{%
  \setlength{\itemsep}{0pt}\setlength{\parskip}{0pt}}
\setcounter{secnumdepth}{5}
% Redefines (sub)paragraphs to behave more like sections
\ifx\paragraph\undefined\else
\let\oldparagraph\paragraph
\renewcommand{\paragraph}[1]{\oldparagraph{#1}\mbox{}}
\fi
\ifx\subparagraph\undefined\else
\let\oldsubparagraph\subparagraph
\renewcommand{\subparagraph}[1]{\oldsubparagraph{#1}\mbox{}}
\fi

%%% Use protect on footnotes to avoid problems with footnotes in titles
\let\rmarkdownfootnote\footnote%
\def\footnote{\protect\rmarkdownfootnote}

%%% Change title format to be more compact
\usepackage{titling}

% Create subtitle command for use in maketitle
\providecommand{\subtitle}[1]{
  \posttitle{
    \begin{center}\large#1\end{center}
    }
}

\setlength{\droptitle}{-2em}

  \title{JAMES manual}
    \pretitle{\vspace{\droptitle}\centering\huge}
  \posttitle{\par}
    \author{Stef van Buuren}
    \preauthor{\centering\large\emph}
  \postauthor{\par}
      \predate{\centering\large\emph}
  \postdate{\par}
    \date{2019-11-11}

\usepackage{booktabs}

\begin{document}
\maketitle

{
\setcounter{tocdepth}{1}
\tableofcontents
}
\hypertarget{prerequisites}{%
\chapter{Prerequisites}\label{prerequisites}}

This is very, very first minimal documentation of JAMES internals.

\hypertarget{intro}{%
\chapter{Introduction}\label{intro}}

Here's an introduction about JAMES

\hypertarget{growth-charts-in-james}{%
\chapter{Growth charts in JAMES}\label{growth-charts-in-james}}

\hypertarget{chart-naming-conventions}{%
\section{Chart naming conventions}\label{chart-naming-conventions}}

The link \url{https://groeidiagrammen.nl/ocpu/lib/james/www/} contains an interactive overview of the available growth charts. There are 342 different charts: for boys and girls, for preterms, for different age ranges, for specific ethnic groups, for height, weight, BMI, and so on. Each chart has a chart code, a character code identifying the design. This section explains the construction of the chart codes.

The GitHub repository \url{https://github.com/stefvanbuuren/chartbox} contains the chart libraries that are available to JAMES. The \texttt{list\_charts()} function produces a tabular overview.

\begin{Shaded}
\begin{Highlighting}[]
\NormalTok{charts <-}\StringTok{ }\NormalTok{chartbox}\OperatorTok{::}\KeywordTok{list_charts}\NormalTok{()}
\KeywordTok{dim}\NormalTok{(charts)}
\end{Highlighting}
\end{Shaded}

\begin{verbatim}
## [1] 342   8
\end{verbatim}

\begin{Shaded}
\begin{Highlighting}[]
\NormalTok{charts[}\KeywordTok{c}\NormalTok{(}\DecValTok{1}\NormalTok{, }\DecValTok{22}\NormalTok{, }\DecValTok{23}\NormalTok{, }\DecValTok{300}\NormalTok{, }\DecValTok{301}\NormalTok{, }\DecValTok{340}\NormalTok{), ]}
\end{Highlighting}
\end{Shaded}

\begin{verbatim}
##     chartgrp chartcode population    sex design  side language week
## 1     nl2010      HJAA         HS   male      A front    dutch     
## 22    nl2010      HMBH         HS female      B   hgt    dutch     
## 23    nl2010      HMBR         HS female      B   wfh    dutch     
## 300  preterm   PMEAN32         PT female      E front    dutch   32
## 301  preterm   PMEAN33         PT female      E front    dutch   33
## 340      who      WMBA    WHOpink female      B front    dutch
\end{verbatim}

The \texttt{chartbox} package currently contains three chart groups: \texttt{nl2010}, \texttt{preterm} and \texttt{who}. Each group collects charts of a similar type.

\begin{longtable}[]{@{}lrlll@{}}
\toprule
Chart Group & Charts & Chart code & Description & Source\tabularnewline
\midrule
\endhead
\texttt{nl2010} & 136 & CCCC & Dutch children 0-21 years, including minorities & \citet{talma2010}\tabularnewline
\texttt{preterm} & 192 & CCCCCNN & Dutch preterms, ga \textless{}= 36 weeks, 0-4 years & \citet{bocca2012}\tabularnewline
\texttt{who} & 14 & CCCC & WHO Child Growth Standards 0-4 years & \href{https://www.who.int/childgrowth/en/}{WHO}\tabularnewline
\bottomrule
\end{longtable}

The chart code is an alpha-numeric code of four (for \texttt{nl2010} and \texttt{who}) or seven (for \texttt{preterm}) that uniquely identifies each of the charts. The table below specifies the full coding schema used to construct the chart codes.

\begin{longtable}[]{@{}clll@{}}
\toprule
Position & Field & Value & Description\tabularnewline
\midrule
\endhead
1 & Population & N & Dutch\tabularnewline
& & T & Turkish\tabularnewline
& & M & Moroccan\tabularnewline
& & H & Hindostan\tabularnewline
& & P & Preterm\tabularnewline
& & W & WHO\tabularnewline
2 & Sex & J & Male\tabularnewline
& & M & Female\tabularnewline
3 & Design & A & 0-15 months\tabularnewline
& & B & 0-4 years, WFH\tabularnewline
& & C & 1-21 years\tabularnewline
& & D & 0-21 years\tabularnewline
& & E & 0-4 years, WFA\tabularnewline
4 & Side & A & A4, front\tabularnewline
& & B & A4, back\tabularnewline
& & C & A4, back, no \texttt{hdc}\tabularnewline
& & H & square, \texttt{hgt}\tabularnewline
& & O & square, \texttt{hdc}\tabularnewline
& & Q & square, \texttt{bmi}\tabularnewline
& & R & square, \texttt{wfh}\tabularnewline
& & W & square, \texttt{wgt}\tabularnewline
& & X & A4, double sided\tabularnewline
5 & Language & N & Dutch\tabularnewline
& & E & English\tabularnewline
6-7 & Week & 25-36 & Gestational age\tabularnewline
\bottomrule
\end{longtable}

For illustration, code \texttt{NJAA} references to Dutch (\texttt{N}), boys (\texttt{J}), 0-15 month (\texttt{A}), front side (\texttt{A}). Likewise, \texttt{PMEAN33} codes for the chart of preterm (\texttt{M}), girls (\texttt{M}), 0-4 years (\texttt{E}), front side (\texttt{A}), Dutch language (\texttt{N}) born at 33 weeks of gestation (\texttt{33}).

Some forms hold multiple growth charts. For example, the \texttt{NJAA} chart is designed for A4 paper size (297mm \(\times\) 210mm) and contains three growth charts: head circumference by age, length by age, and weight by age. Some others have no diagram, like \texttt{NJAB} with explanations. All square formats hold one growth chart. All of the square forms have equal sizes (160mm \(\times\) 160mm).

The following table lists the measures per design-form combination.

\begin{longtable}[]{@{}ccll@{}}
\toprule
Design & Side & Measure & Description\tabularnewline
\midrule
\endhead
A & A & hdc & Head circumference by age, 0-15 mo\tabularnewline
& & hgt & Length by age, 0-15 mo\tabularnewline
& & wgt & Weight by age, 0-15 mo\tabularnewline
& B & &\tabularnewline
& H & hgt & Length by age, 0-15 mo\tabularnewline
& O & hdc & Head circumference by age, 0-15 mo\tabularnewline
& W & wgt & Weight by age, 0-15 mo\tabularnewline
B & A & wfh & Weight for height, 0-4 yr\tabularnewline
& & hgt & Length by age, 0-4 yr\tabularnewline
& B & hdc & Head circumference by age, 0-4 yr\tabularnewline
& C & &\tabularnewline
& H & hgt & Height by age, 0-4 yr\tabularnewline
& O & hdc & Head circumference by age, 0-4 yr\tabularnewline
& R & wfh & Weight for height, 0-4 yr\tabularnewline
& W & wgt & Weight by age, 0-4 yr\tabularnewline
C & A & wfh & Weight for height, 1-21 yr\tabularnewline
& & hgt & height by age, 1-21 yr\tabularnewline
& B & bmi & BMI by age, 1-21 yr\tabularnewline
& & hdc & Head circumference by age, 1-21 yr\tabularnewline
& C & bmi & BMI by age, 1-21 yr\tabularnewline
& H & hgt & Height by age, 1-21 yr\tabularnewline
& O & hdc & Head circumference by age, 1-21 yr\tabularnewline
& Q & bmi & Body mass index by age, 1-21 yr\tabularnewline
& R & wfh & Weight for height, 1-21 yr\tabularnewline
E & A & wgt & Weight by age, 0-4 yr\tabularnewline
& & hgt & height by age, 0-4 yr\tabularnewline
& B & hdc & Head circumference by age, 0-4 yr\tabularnewline
& H & hgt & Height by age, 0-4 yr\tabularnewline
& O & hdc & Head circumference by age, 0-4 yr\tabularnewline
& W & wgt & Weight by age, 0-4 yr\tabularnewline
\bottomrule
\end{longtable}

\hypertarget{methods}{%
\chapter{Methods}\label{methods}}

We describe our methods in this chapter.

\bibliography{book.bib,packages.bib}


\end{document}

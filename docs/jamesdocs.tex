\PassOptionsToPackage{unicode=true}{hyperref} % options for packages loaded elsewhere
\PassOptionsToPackage{hyphens}{url}
%
\documentclass[]{book}
\usepackage{lmodern}
\usepackage{amssymb,amsmath}
\usepackage{ifxetex,ifluatex}
\usepackage{fixltx2e} % provides \textsubscript
\ifnum 0\ifxetex 1\fi\ifluatex 1\fi=0 % if pdftex
  \usepackage[T1]{fontenc}
  \usepackage[utf8]{inputenc}
  \usepackage{textcomp} % provides euro and other symbols
\else % if luatex or xelatex
  \usepackage{unicode-math}
  \defaultfontfeatures{Ligatures=TeX,Scale=MatchLowercase}
\fi
% use upquote if available, for straight quotes in verbatim environments
\IfFileExists{upquote.sty}{\usepackage{upquote}}{}
% use microtype if available
\IfFileExists{microtype.sty}{%
\usepackage[]{microtype}
\UseMicrotypeSet[protrusion]{basicmath} % disable protrusion for tt fonts
}{}
\IfFileExists{parskip.sty}{%
\usepackage{parskip}
}{% else
\setlength{\parindent}{0pt}
\setlength{\parskip}{6pt plus 2pt minus 1pt}
}
\usepackage{hyperref}
\hypersetup{
            pdftitle={JAMES manual},
            pdfauthor={Stef van Buuren},
            pdfborder={0 0 0},
            breaklinks=true}
\urlstyle{same}  % don't use monospace font for urls
\usepackage{color}
\usepackage{fancyvrb}
\newcommand{\VerbBar}{|}
\newcommand{\VERB}{\Verb[commandchars=\\\{\}]}
\DefineVerbatimEnvironment{Highlighting}{Verbatim}{commandchars=\\\{\}}
% Add ',fontsize=\small' for more characters per line
\usepackage{framed}
\definecolor{shadecolor}{RGB}{248,248,248}
\newenvironment{Shaded}{\begin{snugshade}}{\end{snugshade}}
\newcommand{\AlertTok}[1]{\textcolor[rgb]{0.94,0.16,0.16}{#1}}
\newcommand{\AnnotationTok}[1]{\textcolor[rgb]{0.56,0.35,0.01}{\textbf{\textit{#1}}}}
\newcommand{\AttributeTok}[1]{\textcolor[rgb]{0.77,0.63,0.00}{#1}}
\newcommand{\BaseNTok}[1]{\textcolor[rgb]{0.00,0.00,0.81}{#1}}
\newcommand{\BuiltInTok}[1]{#1}
\newcommand{\CharTok}[1]{\textcolor[rgb]{0.31,0.60,0.02}{#1}}
\newcommand{\CommentTok}[1]{\textcolor[rgb]{0.56,0.35,0.01}{\textit{#1}}}
\newcommand{\CommentVarTok}[1]{\textcolor[rgb]{0.56,0.35,0.01}{\textbf{\textit{#1}}}}
\newcommand{\ConstantTok}[1]{\textcolor[rgb]{0.00,0.00,0.00}{#1}}
\newcommand{\ControlFlowTok}[1]{\textcolor[rgb]{0.13,0.29,0.53}{\textbf{#1}}}
\newcommand{\DataTypeTok}[1]{\textcolor[rgb]{0.13,0.29,0.53}{#1}}
\newcommand{\DecValTok}[1]{\textcolor[rgb]{0.00,0.00,0.81}{#1}}
\newcommand{\DocumentationTok}[1]{\textcolor[rgb]{0.56,0.35,0.01}{\textbf{\textit{#1}}}}
\newcommand{\ErrorTok}[1]{\textcolor[rgb]{0.64,0.00,0.00}{\textbf{#1}}}
\newcommand{\ExtensionTok}[1]{#1}
\newcommand{\FloatTok}[1]{\textcolor[rgb]{0.00,0.00,0.81}{#1}}
\newcommand{\FunctionTok}[1]{\textcolor[rgb]{0.00,0.00,0.00}{#1}}
\newcommand{\ImportTok}[1]{#1}
\newcommand{\InformationTok}[1]{\textcolor[rgb]{0.56,0.35,0.01}{\textbf{\textit{#1}}}}
\newcommand{\KeywordTok}[1]{\textcolor[rgb]{0.13,0.29,0.53}{\textbf{#1}}}
\newcommand{\NormalTok}[1]{#1}
\newcommand{\OperatorTok}[1]{\textcolor[rgb]{0.81,0.36,0.00}{\textbf{#1}}}
\newcommand{\OtherTok}[1]{\textcolor[rgb]{0.56,0.35,0.01}{#1}}
\newcommand{\PreprocessorTok}[1]{\textcolor[rgb]{0.56,0.35,0.01}{\textit{#1}}}
\newcommand{\RegionMarkerTok}[1]{#1}
\newcommand{\SpecialCharTok}[1]{\textcolor[rgb]{0.00,0.00,0.00}{#1}}
\newcommand{\SpecialStringTok}[1]{\textcolor[rgb]{0.31,0.60,0.02}{#1}}
\newcommand{\StringTok}[1]{\textcolor[rgb]{0.31,0.60,0.02}{#1}}
\newcommand{\VariableTok}[1]{\textcolor[rgb]{0.00,0.00,0.00}{#1}}
\newcommand{\VerbatimStringTok}[1]{\textcolor[rgb]{0.31,0.60,0.02}{#1}}
\newcommand{\WarningTok}[1]{\textcolor[rgb]{0.56,0.35,0.01}{\textbf{\textit{#1}}}}
\usepackage{longtable,booktabs}
% Fix footnotes in tables (requires footnote package)
\IfFileExists{footnote.sty}{\usepackage{footnote}\makesavenoteenv{longtable}}{}
\usepackage{graphicx,grffile}
\makeatletter
\def\maxwidth{\ifdim\Gin@nat@width>\linewidth\linewidth\else\Gin@nat@width\fi}
\def\maxheight{\ifdim\Gin@nat@height>\textheight\textheight\else\Gin@nat@height\fi}
\makeatother
% Scale images if necessary, so that they will not overflow the page
% margins by default, and it is still possible to overwrite the defaults
% using explicit options in \includegraphics[width, height, ...]{}
\setkeys{Gin}{width=\maxwidth,height=\maxheight,keepaspectratio}
\setlength{\emergencystretch}{3em}  % prevent overfull lines
\providecommand{\tightlist}{%
  \setlength{\itemsep}{0pt}\setlength{\parskip}{0pt}}
\setcounter{secnumdepth}{5}
% Redefines (sub)paragraphs to behave more like sections
\ifx\paragraph\undefined\else
\let\oldparagraph\paragraph
\renewcommand{\paragraph}[1]{\oldparagraph{#1}\mbox{}}
\fi
\ifx\subparagraph\undefined\else
\let\oldsubparagraph\subparagraph
\renewcommand{\subparagraph}[1]{\oldsubparagraph{#1}\mbox{}}
\fi

% set default figure placement to htbp
\makeatletter
\def\fps@figure{htbp}
\makeatother

\usepackage{booktabs}
\usepackage[]{natbib}
\bibliographystyle{apalike}

\title{JAMES manual}
\author{Stef van Buuren}
\date{2020-04-06}

\begin{document}
\maketitle

{
\setcounter{tocdepth}{1}
\tableofcontents
}
\hypertarget{prerequisites}{%
\chapter{Prerequisites}\label{prerequisites}}

This is very, very first minimal documentation of JAMES internals.

\hypertarget{intro}{%
\chapter{Introduction}\label{intro}}

Here's an introduction about JAMES

\hypertarget{james-data-format}{%
\chapter{JAMES data format}\label{james-data-format}}

\hypertarget{objective}{%
\section{Objective}\label{objective}}

This chapter describes the format of the input data accepted by JAMES. The specification

\begin{itemize}
\tightlist
\item
  closely follows the definition of the \href{https://www.ncj.nl/themadossiers/informatisering/basisdataset/}{Basisdataset JGZ 3.25 (2018)};
\item
  defines data objects;
\item
  defines the actions taken by JAMES in case of incorrect, missing or out-of-range data;
\item
  defines the error messages for informing the client.
\end{itemize}

\hypertarget{generic-object-model}{%
\section{Generic object model}\label{generic-object-model}}

\hypertarget{epremdossier-class}{%
\subsection{EPremDossier Class}\label{epremdossier-class}}

\hypertarget{object-model}{%
\subsubsection{Object model}\label{object-model}}

\begin{longtable}[]{@{}lll@{}}
\toprule
EPremDossier & Instance & Class\tabularnewline
\midrule
\endhead
-\textgreater{} & Clientgegevens & EPremGroep\tabularnewline
-\textgreater{} & Contactmomenten & EPremContactmoment\tabularnewline
\bottomrule
\end{longtable}

\hypertarget{syntax-c}{%
\subsubsection{\texorpdfstring{Syntax \texttt{C\#}}{Syntax C\#}}\label{syntax-c}}

\texttt{public\ class\ EPremDossier}

\hypertarget{public-properties}{%
\subsubsection{Public properties}\label{public-properties}}

\begin{longtable}[]{@{}lll@{}}
\toprule
Name & Description & Required\tabularnewline
\midrule
\endhead
Clientgegevens & Class with basic child data & Y\tabularnewline
Contactmomenten & Class with data per visit & N\tabularnewline
InstrumentCode & Integer identifying the instrument & Ignored\tabularnewline
OrganisatieCode & Integer identifying the care organisation & Y\tabularnewline
Referentie & String identifying the request & N\tabularnewline
\bottomrule
\end{longtable}

\hypertarget{epremgroep-class}{%
\subsection{EPremGroep Class}\label{epremgroep-class}}

\hypertarget{object-model-1}{%
\subsubsection{Object model}\label{object-model-1}}

\begin{longtable}[]{@{}lll@{}}
\toprule
EPremGroep & Instance & Class\tabularnewline
\midrule
\endhead
-\textgreater{} & Elementen & EPremElement\tabularnewline
-\textgreater{} & Groepen & EPremGroep\tabularnewline
\bottomrule
\end{longtable}

\hypertarget{syntax-c-1}{%
\subsubsection{\texorpdfstring{Syntax \texttt{C\#}}{Syntax C\#}}\label{syntax-c-1}}

\texttt{public\ class\ EPremGroep}

\hypertarget{public-properties-1}{%
\subsubsection{Public properties}\label{public-properties-1}}

\begin{longtable}[]{@{}lll@{}}
\toprule
Name & Description & Required\tabularnewline
\midrule
\endhead
Elementen & Class with BDS-elements & Y\tabularnewline
Groepen & Class with groups of BDS-elements & N\tabularnewline
\bottomrule
\end{longtable}

\hypertarget{epremelement-class}{%
\subsection{EPremElement Class}\label{epremelement-class}}

\hypertarget{syntax-c-2}{%
\subsubsection{\texorpdfstring{Syntax \texttt{C\#}}{Syntax C\#}}\label{syntax-c-2}}

\texttt{public\ class\ EPremElement}

\hypertarget{public-properties-2}{%
\subsubsection{Public properties}\label{public-properties-2}}

\begin{longtable}[]{@{}lll@{}}
\toprule
Name & Description & Required\tabularnewline
\midrule
\endhead
Bdsnummer & Integer identifying the BDS-field & Y\tabularnewline
InternNummer & Integer identifying internal field & Ignored\tabularnewline
Waarde & Value of the BDS-field & Y\tabularnewline
Waardeomschrijving & Descriptive label for value & Ignored\tabularnewline
\bottomrule
\end{longtable}

\hypertarget{epremcontactmoment-class}{%
\subsection{EPremContactmoment Class}\label{epremcontactmoment-class}}

\hypertarget{object-model-2}{%
\subsubsection{Object model}\label{object-model-2}}

\begin{longtable}[]{@{}lll@{}}
\toprule
EPremContactmoment & Instance & Class\tabularnewline
\midrule
\endhead
-\textgreater{} & Elementen & EPremElement\tabularnewline
-\textgreater{} & Groepen & EPremGroep\tabularnewline
\bottomrule
\end{longtable}

\hypertarget{syntax-c-3}{%
\subsubsection{\texorpdfstring{Syntax \texttt{C\#}}{Syntax C\#}}\label{syntax-c-3}}

\texttt{public\ class\ EPremContactmoment\ :\ EPremGroep}

\hypertarget{public-properties-3}{%
\subsubsection{Public properties}\label{public-properties-3}}

\begin{longtable}[]{@{}lll@{}}
\toprule
Name & Description & Required\tabularnewline
\midrule
\endhead
Elementen & Class with BDS-elements & Y\tabularnewline
Groepen & Class with groups of BDS-elements & N\tabularnewline
Tijdstip & Date of visit & Y\tabularnewline
\bottomrule
\end{longtable}

\hypertarget{bds-elements}{%
\section{BDS-elements}\label{bds-elements}}

\begin{longtable}[]{@{}rlrll@{}}
\toprule
BDS & Description & Value & Label & Required\tabularnewline
\midrule
\endhead
19 & Sex of child & ``0'' & Unknown & Y\tabularnewline
& & ``1'' & Male &\tabularnewline
& & ``2'' & Female &\tabularnewline
& & ``3'' & Not specified &\tabularnewline
20 & Date of birth & ``yyyymmdd'' & year-month-day & Y\tabularnewline
62 & Caretaker relation & ``01'' & biological father & N\tabularnewline
& & ``02'' & biological mother &\tabularnewline
& & ``03'' & male partner, stepfather &\tabularnewline
& & ``04'' & female partner, stepmother &\tabularnewline
& & ``05'' & adoptive father &\tabularnewline
& & ``06'' & adoptive mother &\tabularnewline
& & ``07'' & foster father &\tabularnewline
& & ``08'' & foster mother &\tabularnewline
& & ``98'' & other &\tabularnewline
63 & Caretaker date of birth & ``yyyymmdd'' & year-month-day & N\tabularnewline
66 & Caretaker education & ``01'' & no primary school & N\tabularnewline
& & ``02'' & primary school, special ed &\tabularnewline
& & ``03'' & VSO-MLK/IVBO/VMBO-LWOO &\tabularnewline
& & ``04'' & LBO/VBO/VMBO-BBL\&KBL &\tabularnewline
& & ``05'' & MAVO/VMBO-GL\&TL &\tabularnewline
& & ``06'' & MBO &\tabularnewline
& & ``07'' & HAVO/VWO &\tabularnewline
& & ``08'' & HBO/HTS/HEAO &\tabularnewline
& & ``09'' & WO &\tabularnewline
& & ``98'' & Other &\tabularnewline
& & ``00'' & Unknown &\tabularnewline
71 & Caretaker birth country & ``dddd'' & \href{https://publicaties.rvig.nl/Landelijke_tabellen/Landelijke_tabellen_32_t_m_60_excl_tabel_35/Landelijke_Tabellen_32_t_m_60_in_csv_formaat}{4-digit code, Table 34} & N\tabularnewline
82 & Gestational age & ``ddd'' & in days & N\tabularnewline
91 & Smoking during pregnancy & ``1'' & yes & N\tabularnewline
& & ``2'' & no &\tabularnewline
& & ``99'' & unknown &\tabularnewline
110 & Birth weight & ``dddd'' & 3-4 digits, grammes & N\tabularnewline
235 & Length/height & ``dddd'' & 3-4 digits, millimeters & N\tabularnewline
245 & Body weight & ``dddddd'' & 3-6 digits, grammes & N\tabularnewline
252 & Head circumference & ``ddd'' & 2-3 digits, millimeters & N\tabularnewline
238 & Height biological mother & ``dddd'' & 3-4 digits, millimeters & N\tabularnewline
240 & Height biological father & ``dddd'' & 3-4 digits, millimeters & N\tabularnewline
510 & Passive smoking & ``01'' & No smoking in house & N\tabularnewline
& & ``02'' & Never with child &\tabularnewline
& & ``03'' & Not in last 7 days &\tabularnewline
& & ``04'' & Yes &\tabularnewline
\bottomrule
\end{longtable}

\hypertarget{error-checking}{%
\section{Error checking}\label{error-checking}}

Error checking of the JSON data occurs in three phases:

\begin{enumerate}
\def\labelenumi{\arabic{enumi}.}
\item
  PHASE 1: Check whether the JSON data are valid JSON. The process terminates
  with an error message if the input JSON is not valid.
\item
  PHASE 2: Validate the JSON data against the JSON schema specification. The process terminates
  with an error if any required fields are missing. The process generates messages for data points
  that do not conform to the JSON schema, but continues.
\item
  PHASE 3: Check the range of the numeric data. The process generates messages for out-of-range
  values, but continues using the specified values.
\end{enumerate}

The default JSON schema in phase 2 is the built-in JSON schema \texttt{bds\_schema\_str.json}, a data format implementing a version that accepts strings as values for BDS-elements.

\hypertarget{growth-charts-in-james}{%
\chapter{Growth charts in JAMES}\label{growth-charts-in-james}}

\hypertarget{chart-naming-conventions}{%
\section{Chart naming conventions}\label{chart-naming-conventions}}

The link \url{https://groeidiagrammen.nl/ocpu/lib/james/www/} contains an interactive overview of the available growth charts. There are 342 different charts: for boys and girls, for preterms, for different age ranges, for specific ethnic groups, for height, weight, BMI, and so on. Each chart has a chart code, a character code identifying the design. This section explains the construction of the chart codes.

The GitHub repository \url{https://github.com/stefvanbuuren/chartbox} contains the chart libraries that are available to JAMES. The \texttt{list\_charts()} function produces a tabular overview.

\begin{Shaded}
\begin{Highlighting}[]
\NormalTok{charts <-}\StringTok{ }\NormalTok{chartbox}\OperatorTok{::}\KeywordTok{list_charts}\NormalTok{()}
\KeywordTok{dim}\NormalTok{(charts)}
\end{Highlighting}
\end{Shaded}

\begin{verbatim}
## [1] 394   8
\end{verbatim}

\begin{Shaded}
\begin{Highlighting}[]
\NormalTok{charts[}\KeywordTok{c}\NormalTok{(}\DecValTok{1}\NormalTok{, }\DecValTok{22}\NormalTok{, }\DecValTok{23}\NormalTok{, }\DecValTok{300}\NormalTok{, }\DecValTok{301}\NormalTok{, }\DecValTok{340}\NormalTok{), ]}
\end{Highlighting}
\end{Shaded}

\begin{verbatim}
##     chartgrp chartcode population    sex design  side language week
## 1     nl2010      HJAA         HS   male      A front    dutch     
## 22    nl2010      HMBH         HS female      B   hgt    dutch     
## 23    nl2010      HMBR         HS female      B   wfh    dutch     
## 300  preterm   PMAHN28         PT female      A   hgt    dutch   28
## 301  preterm   PMAHN29         PT female      A   hgt    dutch   29
## 340  preterm   PMEAN32         PT female      E front    dutch   32
\end{verbatim}

The \texttt{chartbox} package currently contains three chart groups: \texttt{nl2010}, \texttt{preterm} and \texttt{who}. Each group collects charts of a similar type.

\begin{longtable}[]{@{}lrlll@{}}
\toprule
Chart Group & Charts & Chart code & Description & Source\tabularnewline
\midrule
\endhead
\texttt{nl2010} & 136 & CCCC & Dutch children 0-21 years, including minorities & \citet{talma2010}\tabularnewline
\texttt{preterm} & 192 & CCCCCNN & Dutch preterms, ga \textless{}= 36 weeks, 0-4 years & \citet{bocca2012}\tabularnewline
\texttt{who} & 14 & CCCC & WHO Child Growth Standards 0-4 years & \href{https://www.who.int/childgrowth/en/}{WHO}\tabularnewline
\bottomrule
\end{longtable}

The chart code is an alpha-numeric code of four (for \texttt{nl2010} and \texttt{who}) or seven (for \texttt{preterm}) that uniquely identifies each of the charts. The table below specifies the full coding schema used to construct the chart codes.

\begin{longtable}[]{@{}clll@{}}
\toprule
Position & Field & Value & Description\tabularnewline
\midrule
\endhead
1 & Population & N & Dutch\tabularnewline
& & T & Turkish\tabularnewline
& & M & Moroccan\tabularnewline
& & H & Hindostan\tabularnewline
& & P & Preterm\tabularnewline
& & W & WHO\tabularnewline
2 & Sex & J & Male\tabularnewline
& & M & Female\tabularnewline
3 & Design & A & 0-15 months\tabularnewline
& & B & 0-4 years, WFH\tabularnewline
& & C & 1-21 years\tabularnewline
& & D & 0-21 years\tabularnewline
& & E & 0-4 years, WFA\tabularnewline
4 & Side & A & A4, front\tabularnewline
& & B & A4, back\tabularnewline
& & C & A4, back, no \texttt{hdc}\tabularnewline
& & H & square, \texttt{hgt}\tabularnewline
& & O & square, \texttt{hdc}\tabularnewline
& & Q & square, \texttt{bmi}\tabularnewline
& & R & square, \texttt{wfh}\tabularnewline
& & W & square, \texttt{wgt}\tabularnewline
& & X & A4, double sided\tabularnewline
5 & Language & N & Dutch\tabularnewline
& & E & English\tabularnewline
6-7 & Week & 25-36 & Gestational age\tabularnewline
\bottomrule
\end{longtable}

For illustration, code \texttt{NJAA} references to Dutch (\texttt{N}), boys (\texttt{J}), 0-15 month (\texttt{A}), front side (\texttt{A}). Likewise, \texttt{PMEAN33} codes for the chart of preterm (\texttt{M}), girls (\texttt{M}), 0-4 years (\texttt{E}), front side (\texttt{A}), Dutch language (\texttt{N}) born at 33 weeks of gestation (\texttt{33}).

Some forms hold multiple growth charts. For example, the \texttt{NJAA} chart is designed for A4 paper size (297mm \(\times\) 210mm) and contains three growth charts: head circumference by age, length by age, and weight by age. Some others have no diagram, like \texttt{NJAB} with explanations. All square formats hold one growth chart. All of the square forms have equal sizes (160mm \(\times\) 160mm).

The following table lists the measures per design-form combination.

\begin{longtable}[]{@{}ccll@{}}
\toprule
Design & Side & Measure & Description\tabularnewline
\midrule
\endhead
A & A & \texttt{hdc} & Head circumference by age, 0-15 mo\tabularnewline
& & \texttt{hgt} & Length by age, 0-15 mo\tabularnewline
& & \texttt{wgt} & Weight by age, 0-15 mo\tabularnewline
& B & &\tabularnewline
& H & \texttt{hgt} & Length by age, 0-15 mo\tabularnewline
& O & \texttt{hdc} & Head circumference by age, 0-15 mo\tabularnewline
& W & \texttt{wgt} & Weight by age, 0-15 mo\tabularnewline
B & A & \texttt{wfh} & Weight for height, 0-4 yr\tabularnewline
& & \texttt{hgt} & Length by age, 0-4 yr\tabularnewline
& B & \texttt{hdc} & Head circumference by age, 0-4 yr\tabularnewline
& C & &\tabularnewline
& H & \texttt{hgt} & Height by age, 0-4 yr\tabularnewline
& O & \texttt{hdc} & Head circumference by age, 0-4 yr\tabularnewline
& R & \texttt{wfh} & Weight for height, 0-4 yr\tabularnewline
& W & \texttt{wgt} & Weight by age, 0-4 yr\tabularnewline
C & A & \texttt{wfh} & Weight for height, 1-21 yr\tabularnewline
& & \texttt{hgt} & height by age, 1-21 yr\tabularnewline
& B & \texttt{bmi} & BMI by age, 1-21 yr\tabularnewline
& & \texttt{hdc} & Head circumference by age, 1-21 yr\tabularnewline
& C & \texttt{bmi} & BMI by age, 1-21 yr\tabularnewline
& H & \texttt{hgt} & Height by age, 1-21 yr\tabularnewline
& O & \texttt{hdc} & Head circumference by age, 1-21 yr\tabularnewline
& Q & \texttt{bmi} & Body mass index by age, 1-21 yr\tabularnewline
& R & \texttt{wfh} & Weight for height, 1-21 yr\tabularnewline
E & A & \texttt{wgt} & Weight by age, 0-4 yr\tabularnewline
& & \texttt{hgt} & height by age, 0-4 yr\tabularnewline
& B & \texttt{hdc} & Head circumference by age, 0-4 yr\tabularnewline
& H & \texttt{hgt} & Height by age, 0-4 yr\tabularnewline
& O & \texttt{hdc} & Head circumference by age, 0-4 yr\tabularnewline
& W & \texttt{wgt} & Weight by age, 0-4 yr\tabularnewline
\bottomrule
\end{longtable}

\hypertarget{methods}{%
\chapter{Methods}\label{methods}}

We describe our methods in this chapter.

\hypertarget{d-score-implementation}{%
\chapter{D-score implementation}\label{d-score-implementation}}

This document describes the actions needed to implement the
\href{https://stefvanbuuren.name/dbook1/}{D-score} into JAMES. The functionality of JAMES is distributed over multiple packages. This set of actions may be of interest when implementing new features.

\hypertarget{actions}{%
\section{Actions}\label{actions}}

\begin{longtable}[]{@{}lll@{}}
\toprule
\begin{minipage}[b]{0.26\columnwidth}\raggedright
Package\strut
\end{minipage} & \begin{minipage}[b]{0.29\columnwidth}\raggedright
PR\strut
\end{minipage} & \begin{minipage}[b]{0.36\columnwidth}\raggedright
Description\strut
\end{minipage}\tabularnewline
\midrule
\endhead
\begin{minipage}[t]{0.26\columnwidth}\raggedright
minihealth\strut
\end{minipage} & \begin{minipage}[t]{0.29\columnwidth}\raggedright
\href{https://github.com/stefvanbuuren/minihealth/commit/03a32f1960e81a685bb749911e6ea297684ab4dc}{03a32f1}\strut
\end{minipage} & \begin{minipage}[t]{0.36\columnwidth}\raggedright
Create milestones descriptions\strut
\end{minipage}\tabularnewline
\begin{minipage}[t]{0.26\columnwidth}\raggedright
dscore\strut
\end{minipage} & \begin{minipage}[t]{0.29\columnwidth}\raggedright
\href{https://github.com/stefvanbuuren/dscore/commit/f0013ce9a02d34ad25dd5c101c6a7c5b1444b53b}{f0013ce}\strut
\end{minipage} & \begin{minipage}[t]{0.36\columnwidth}\raggedright
Link BDS number to Van Wiechen milestones\strut
\end{minipage}\tabularnewline
\begin{minipage}[t]{0.26\columnwidth}\raggedright
dscore\strut
\end{minipage} & \begin{minipage}[t]{0.29\columnwidth}\raggedright
\href{https://github.com/stefvanbuuren/dscore/commit/688685477082ac6c040b9d18035b5178a39a5cc0}{6886854}\strut
\end{minipage} & \begin{minipage}[t]{0.36\columnwidth}\raggedright
Fine tuning of milestone labels\strut
\end{minipage}\tabularnewline
\begin{minipage}[t]{0.26\columnwidth}\raggedright
minihealth\strut
\end{minipage} & \begin{minipage}[t]{0.29\columnwidth}\raggedright
\strut
\end{minipage} & \begin{minipage}[t]{0.36\columnwidth}\raggedright
Create the \texttt{bds\_lexicon} object\strut
\end{minipage}\tabularnewline
\begin{minipage}[t]{0.26\columnwidth}\raggedright
minihealth\strut
\end{minipage} & \begin{minipage}[t]{0.29\columnwidth}\raggedright
\href{https://github.com/stefvanbuuren/minihealth/commit/4893982631717539b070c8d19b88b2b10319f2ee}{4893982}\strut
\end{minipage} & \begin{minipage}[t]{0.36\columnwidth}\raggedright
Add milestones to BDS validation JSON schema\strut
\end{minipage}\tabularnewline
\begin{minipage}[t]{0.26\columnwidth}\raggedright
minihealth\strut
\end{minipage} & \begin{minipage}[t]{0.29\columnwidth}\raggedright
\href{https://github.com/stefvanbuuren/minihealth/commit/0069671c8d09f64f2436faa93b764bd288324fcc}{0069671}\strut
\end{minipage} & \begin{minipage}[t]{0.36\columnwidth}\raggedright
Add \texttt{convert\_ddi\_gsed()} to convert BDS-milestones into GSED items\strut
\end{minipage}\tabularnewline
\begin{minipage}[t]{0.26\columnwidth}\raggedright
minihealth\strut
\end{minipage} & \begin{minipage}[t]{0.29\columnwidth}\raggedright
\href{https://github.com/stefvanbuuren/minihealth/commit/8ab1392fdbe781adfa004fcadb4f661c14487cf2}{8ab1392}\strut
\end{minipage} & \begin{minipage}[t]{0.36\columnwidth}\raggedright
Add a new class \texttt{individualDS} for storing milestones, D-score and DAZ\strut
\end{minipage}\tabularnewline
\begin{minipage}[t]{0.26\columnwidth}\raggedright
clopus\strut
\end{minipage} & \begin{minipage}[t]{0.29\columnwidth}\raggedright
\href{https://github.com/stefvanbuuren/clopus/commit/1182cb02508a3207c0c9bcb35232851c38d24179}{1182cb0}\strut
\end{minipage} & \begin{minipage}[t]{0.36\columnwidth}\raggedright
Add Dutch and GCDG D-score references\strut
\end{minipage}\tabularnewline
\begin{minipage}[t]{0.26\columnwidth}\raggedright
clopus\strut
\end{minipage} & \begin{minipage}[t]{0.29\columnwidth}\raggedright
\href{https://github.com/stefvanbuuren/clopus/commit/7bdbcd9629bb215f466bbc2e2f886b6f16c5b5e9}{7bdbcd9}\strut
\end{minipage} & \begin{minipage}[t]{0.36\columnwidth}\raggedright
Construct age-shifted D-score references for preterms\strut
\end{minipage}\tabularnewline
\begin{minipage}[t]{0.26\columnwidth}\raggedright
clopus\strut
\end{minipage} & \begin{minipage}[t]{0.29\columnwidth}\raggedright
\href{https://github.com/stefvanbuuren/clopus/commit/ceab7f9547b9ae843a97b640d920ea0c36185053}{ceab7f9}\strut
\end{minipage} & \begin{minipage}[t]{0.36\columnwidth}\raggedright
Import the D-score references into \texttt{clopus}\strut
\end{minipage}\tabularnewline
\begin{minipage}[t]{0.26\columnwidth}\raggedright
chartdesigner\strut
\end{minipage} & \begin{minipage}[t]{0.29\columnwidth}\raggedright
\href{https://github.com/stefvanbuuren/chartdesigner/commit/68831906cef5cdca0d8851b057a01cc8be4fff6d}{6883190}\strut
\end{minipage} & \begin{minipage}[t]{0.36\columnwidth}\raggedright
Add chart constructor functions for D-score, both terms and pre-terms\strut
\end{minipage}\tabularnewline
\begin{minipage}[t]{0.26\columnwidth}\raggedright
chartdesigner\strut
\end{minipage} & \begin{minipage}[t]{0.29\columnwidth}\raggedright
\href{https://github.com/stefvanbuuren/chartdesigner/commit/511f456884bc62d6bebc9b5ef33cebb940dc126d}{511f456}\strut
\end{minipage} & \begin{minipage}[t]{0.36\columnwidth}\raggedright
Extend internal \texttt{set.axes.design()} to D-score charts\strut
\end{minipage}\tabularnewline
\begin{minipage}[t]{0.26\columnwidth}\raggedright
chartdesigner\strut
\end{minipage} & \begin{minipage}[t]{0.29\columnwidth}\raggedright
\href{https://github.com/stefvanbuuren/chartdesigner/commit/6582af8753331a25d8970683d4523c69d6959f0d}{6582af8}\strut
\end{minipage} & \begin{minipage}[t]{0.36\columnwidth}\raggedright
Extend to \texttt{axes.locations} object to D-score charts\strut
\end{minipage}\tabularnewline
\begin{minipage}[t]{0.26\columnwidth}\raggedright
chartdesigner\strut
\end{minipage} & \begin{minipage}[t]{0.29\columnwidth}\raggedright
\href{https://github.com/stefvanbuuren/chartdesigner/commit/47e3cc39d1253a41d3cc81d3d6dd55bffa842f18}{47e3cc3}\strut
\end{minipage} & \begin{minipage}[t]{0.36\columnwidth}\raggedright
Create \texttt{dchart()} function and extend its helper functions\strut
\end{minipage}\tabularnewline
\begin{minipage}[t]{0.26\columnwidth}\raggedright
chartdesigner\strut
\end{minipage} & \begin{minipage}[t]{0.29\columnwidth}\raggedright
\href{https://github.com/stefvanbuuren/chartdesigner/commit/fbbc7c8647ea010b2292fa9dd9b253ce95b6a54b}{fbbc7c8}\strut
\end{minipage} & \begin{minipage}[t]{0.36\columnwidth}\raggedright
Function \texttt{chartcode()} factory, make one function for each chart code\strut
\end{minipage}\tabularnewline
\begin{minipage}[t]{0.26\columnwidth}\raggedright
chartcatalog\strut
\end{minipage} & \begin{minipage}[t]{0.29\columnwidth}\raggedright
\href{https://github.com/stefvanbuuren/chartcatalog/commit/cc467888dd5346d7ed2c0a78b976a8fa818f712c}{cc46788}\strut
\end{minipage} & \begin{minipage}[t]{0.36\columnwidth}\raggedright
Extend the chart naming system to D-score charts\strut
\end{minipage}\tabularnewline
\begin{minipage}[t]{0.26\columnwidth}\raggedright
chartcatalog\strut
\end{minipage} & \begin{minipage}[t]{0.29\columnwidth}\raggedright
\href{https://github.com/stefvanbuuren/chartcatalog/commit/84aaded206a5050660bd347d2e093af93b6b7ae1}{84aaded}\strut
\end{minipage} & \begin{minipage}[t]{0.36\columnwidth}\raggedright
Extend the lookup table \texttt{ynames\_lookup} to handle new D-score charts\strut
\end{minipage}\tabularnewline
\begin{minipage}[t]{0.26\columnwidth}\raggedright
chartbox\strut
\end{minipage} & \begin{minipage}[t]{0.29\columnwidth}\raggedright
\href{https://github.com/stefvanbuuren/chartbox/commit/aa310672a717f9777e2daac409d5ec40f3db509f}{aa31067}\strut
\end{minipage} & \begin{minipage}[t]{0.36\columnwidth}\raggedright
Extend chart box with all D-score charts\strut
\end{minipage}\tabularnewline
\begin{minipage}[t]{0.26\columnwidth}\raggedright
james\strut
\end{minipage} & \begin{minipage}[t]{0.29\columnwidth}\raggedright
\href{https://github.com/stefvanbuuren/james/commits/dscore}{6412840}\strut
\end{minipage} & \begin{minipage}[t]{0.36\columnwidth}\raggedright
Add radio button for D-score charts\strut
\end{minipage}\tabularnewline
\begin{minipage}[t]{0.26\columnwidth}\raggedright
minihealth\strut
\end{minipage} & \begin{minipage}[t]{0.29\columnwidth}\raggedright
\href{https://github.com/stefvanbuuren/minihealth/commit/06a04c9ce70546db7998de5147a15897af0c7ddb}{06a04c9}\strut
\end{minipage} & \begin{minipage}[t]{0.36\columnwidth}\raggedright
Calculate D-score and DAZ\strut
\end{minipage}\tabularnewline
\begin{minipage}[t]{0.26\columnwidth}\raggedright
chartplotter\strut
\end{minipage} & \begin{minipage}[t]{0.29\columnwidth}\raggedright
\href{https://github.com/stefvanbuuren/chartplotter/commit/4b5863813da5304ab5117feba216b7a0822fcd16}{4b58638}\strut
\end{minipage} & \begin{minipage}[t]{0.36\columnwidth}\raggedright
Skip the \texttt{dsc} field for finding matches\strut
\end{minipage}\tabularnewline
\begin{minipage}[t]{0.26\columnwidth}\raggedright
minihealth\strut
\end{minipage} & \begin{minipage}[t]{0.29\columnwidth}\raggedright
\href{https://github.com/stefvanbuuren/minihealth/commit/816be33b8921ab8d80d8a750d5c4e11966a58c18}{816be33}\strut
\end{minipage} & \begin{minipage}[t]{0.36\columnwidth}\raggedright
Add D-score and DAZ to class \texttt{individualAN}\strut
\end{minipage}\tabularnewline
\begin{minipage}[t]{0.26\columnwidth}\raggedright
donordata\strut
\end{minipage} & \begin{minipage}[t]{0.29\columnwidth}\raggedright
\href{https://github.com/stefvanbuuren/donordata/commit/77e01b40e2be75c19f27423092aa5626b0d5d3c3}{77e01b4}\strut
\end{minipage} & \begin{minipage}[t]{0.36\columnwidth}\raggedright
Add milestones to SMOCC donor data\strut
\end{minipage}\tabularnewline
\begin{minipage}[t]{0.26\columnwidth}\raggedright
donordata\strut
\end{minipage} & \begin{minipage}[t]{0.29\columnwidth}\raggedright
\href{https://github.com/stefvanbuuren/donordata/commit/ecb3413b115c3193135b1af118597768b083dc41}{ecb3413}\strut
\end{minipage} & \begin{minipage}[t]{0.36\columnwidth}\raggedright
Calculate D-score and DAZ for SMOCC data\strut
\end{minipage}\tabularnewline
\begin{minipage}[t]{0.26\columnwidth}\raggedright
donordata\strut
\end{minipage} & \begin{minipage}[t]{0.29\columnwidth}\raggedright
\href{https://github.com/stefvanbuuren/donordata/commit/3fa9d4de347d09ccff32a887cf6bd8dadfdfaeb6}{3fa9d4d}\strut
\end{minipage} & \begin{minipage}[t]{0.36\columnwidth}\raggedright
Fit and store brokenstick model for D-score on SMOCC data\strut
\end{minipage}\tabularnewline
\begin{minipage}[t]{0.26\columnwidth}\raggedright
donorloader\strut
\end{minipage} & \begin{minipage}[t]{0.29\columnwidth}\raggedright
\href{https://github.com/stefvanbuuren/donorloader/commit/c22c44607b40ff6bda2009c25894362b68831275}{c22c446}\strut
\end{minipage} & \begin{minipage}[t]{0.36\columnwidth}\raggedright
Update internal data after changes in donordata\strut
\end{minipage}\tabularnewline
\begin{minipage}[t]{0.26\columnwidth}\raggedright
jamesdocs\strut
\end{minipage} & \begin{minipage}[t]{0.29\columnwidth}\raggedright
TBD\strut
\end{minipage} & \begin{minipage}[t]{0.36\columnwidth}\raggedright
Document steps (this file)\strut
\end{minipage}\tabularnewline
\begin{minipage}[t]{0.26\columnwidth}\raggedright
donordata\strut
\end{minipage} & \begin{minipage}[t]{0.29\columnwidth}\raggedright
\href{https://github.com/stefvanbuuren/donordata/commit/7983c3d81a69eb9c6b81a3e0f0668e892c5abf5a}{7983c3}\strut
\end{minipage} & \begin{minipage}[t]{0.36\columnwidth}\raggedright
Saves the item scores to create JSON files\strut
\end{minipage}\tabularnewline
\begin{minipage}[t]{0.26\columnwidth}\raggedright
donordata\strut
\end{minipage} & \begin{minipage}[t]{0.29\columnwidth}\raggedright
\href{https://github.com/stefvanbuuren/donordata/commit/1537182a0df4180e0000d59225c6355a9c506472}{1537182}\strut
\end{minipage} & \begin{minipage}[t]{0.36\columnwidth}\raggedright
Save mapping between SMOCC and BDS coding scheme\strut
\end{minipage}\tabularnewline
\begin{minipage}[t]{0.26\columnwidth}\raggedright
donorloader\strut
\end{minipage} & \begin{minipage}[t]{0.29\columnwidth}\raggedright
\href{https://github.com/stefvanbuuren/donorloader/commit/e9a8ed9054ebe1ba3a060883d53cfa0ce1f963d2}{e9a8ed}\strut
\end{minipage} & \begin{minipage}[t]{0.36\columnwidth}\raggedright
Make \texttt{smocc\_bds} available to JAMES\strut
\end{minipage}\tabularnewline
\begin{minipage}[t]{0.26\columnwidth}\raggedright
jamestest\strut
\end{minipage} & \begin{minipage}[t]{0.29\columnwidth}\raggedright
\href{https://github.com/stefvanbuuren/jamestest/commit/6484191fd8de0d93b35354aa3ee846447aa81df6}{648419}\strut
\end{minipage} & \begin{minipage}[t]{0.36\columnwidth}\raggedright
Regenerate smocc JSON files to include DDI scores\strut
\end{minipage}\tabularnewline
\begin{minipage}[t]{0.26\columnwidth}\raggedright
jamestest\strut
\end{minipage} & \begin{minipage}[t]{0.29\columnwidth}\raggedright
\href{https://github.com/stefvanbuuren/jamestest/commit/ce1dbe5591ca7182fef05ea6c5cb0d1361e876dc}{ce1dbe}\strut
\end{minipage} & \begin{minipage}[t]{0.36\columnwidth}\raggedright
Update the \texttt{installed.cabinets} object with the new individual milestones data\strut
\end{minipage}\tabularnewline
\begin{minipage}[t]{0.26\columnwidth}\raggedright
minihealth\strut
\end{minipage} & \begin{minipage}[t]{0.29\columnwidth}\raggedright
\href{https://github.com/stefvanbuuren/minihealth/commit/4dda8daabca8811a4cc321c66aaccf31c4aba83f}{4dda8d}\strut
\end{minipage} & \begin{minipage}[t]{0.36\columnwidth}\raggedright
Add class \texttt{individualRW} to store and convert raw milestones data\strut
\end{minipage}\tabularnewline
\begin{minipage}[t]{0.26\columnwidth}\raggedright
minihealth\strut
\end{minipage} & \begin{minipage}[t]{0.29\columnwidth}\raggedright
\href{https://github.com/stefvanbuuren/minihealth/commit/9e03e7e39b007a9687d21efbaa13e77283866d5c}{9e03e7}\strut
\end{minipage} & \begin{minipage}[t]{0.36\columnwidth}\raggedright
Complete the JSON validator schema\strut
\end{minipage}\tabularnewline
\bottomrule
\end{longtable}

\hypertarget{dockerfile-for-james}{%
\chapter{Dockerfile for JAMES}\label{dockerfile-for-james}}

\hypertarget{objective-1}{%
\section{Objective}\label{objective-1}}

This chapter describes how to build and deploy JAMES as a Docker container.

\hypertarget{pre-requisites}{%
\section{Pre-requisites}\label{pre-requisites}}

JAMES is currently constructed from a collection of \texttt{R} packages. The top-level package at \url{https://github.com/stefvanbuuren/james} also defines a Javascript interface in the \texttt{inst/www} directory. Deployment of JAMES relies on the \texttt{OpenCPU} server. In principle, it is enough to install the \texttt{james} package on the \texttt{OpenCPU} server, and will also install all dependencies.

The following is needed to build and run a JAMES image:

\begin{itemize}
\item
  Permission to read from the following private repo's:

  \begin{itemize}
  \tightlist
  \item
    \texttt{stefvanbuuren/chartplotter}
  \item
    \texttt{stefvanbuuren/clopus}
  \item
    \texttt{stefvanbuuren/curvematching}
  \item
    \texttt{stefvanbuuren/donorloader}
  \end{itemize}
\item
  Personal Github token with repo scope from \href{https://help.github.com/en/github/authenticating-to-github/creating-a-personal-access-token-for-the-command-line}{here}, Generate a token with only scope repo.
\item
  Install \texttt{Docker\ Desktop} on your local machine, and run some tutorials
\end{itemize}

\hypertarget{dockerfile}{%
\section{Dockerfile}\label{dockerfile}}

\begin{Shaded}
\begin{Highlighting}[]
\ExtensionTok{FROM}\NormalTok{ opencpu/rstudio as intermediate}

\CommentTok{# install for V8 package}
\ExtensionTok{RUN}\NormalTok{ apt-get update }\KeywordTok{&&} \ExtensionTok{apt-get}\NormalTok{ install -y }\DataTypeTok{\textbackslash{} }
    \ExtensionTok{libv8-dev}

\CommentTok{# repo_token.txt should be something like "GITHUB_PAT=624adeaa..."}
\CommentTok{# to be able to install packages from private repo's}
\ExtensionTok{COPY}\NormalTok{ repo_token.txt .Renviron}

\CommentTok{# install R packages needed for JAMES}
\ExtensionTok{RUN}\NormalTok{ \textbackslash{}}
\NormalTok{    R -e }\StringTok{'install.packages("remotes")'}\NormalTok{ \textbackslash{}}
\NormalTok{    R -e }\StringTok{'remotes::install_github("stefvanbuuren/james")'}

\CommentTok{# rebuild the layer}
\ExtensionTok{FROM}\NormalTok{ opencpu/rstudio}

\ExtensionTok{LABEL}\NormalTok{ maintainer=}\StringTok{"stef.vanbuuren@tno.nl"} 

\CommentTok{# re-install for V8 package}
\ExtensionTok{RUN}\NormalTok{ apt-get update }\KeywordTok{&&} \ExtensionTok{apt-get}\NormalTok{ install -y }\DataTypeTok{\textbackslash{} }
    \ExtensionTok{libv8-dev}

\ExtensionTok{COPY}\NormalTok{ --from=intermediate /usr/local/lib/R/site-library /usr/local/lib/R/site-library}

\ExtensionTok{CMD}\NormalTok{ service cron start }\KeywordTok{&&} \ExtensionTok{/usr/lib/rstudio-server/bin/rserver} \KeywordTok{&&} \ExtensionTok{apachectl}\NormalTok{ -DFOREGROUND}
\end{Highlighting}
\end{Shaded}

\begin{itemize}
\tightlist
\item
  Create a fresh directory, and \texttt{cd} to this directory in your terminal
\item
  Store the script given above in a file named \texttt{Dockerfile}
\item
  Create a file named \texttt{repo\_token.txt} with one line: \texttt{GITHUB\_PAT=624adeaa...} with your token
\end{itemize}

\hypertarget{docker-commands}{%
\section{Docker commands}\label{docker-commands}}

Build the \texttt{james} image, type in a terminal

\begin{Shaded}
\begin{Highlighting}[]
\ExtensionTok{docker}\NormalTok{ build -t james .}
\end{Highlighting}
\end{Shaded}

This may takes a long time (30-60 minutes), in which the entire application is downloaded from various web-locations. After (hopefully succesful) completion, check the image

\begin{Shaded}
\begin{Highlighting}[]
\ExtensionTok{docker}\NormalTok{ images}
\end{Highlighting}
\end{Shaded}

If all is well, the top line is called \texttt{james}. Now run the container on your local machine:

\begin{Shaded}
\begin{Highlighting}[]
\ExtensionTok{docker}\NormalTok{ run -t -d -p 80:80 -p 8004:8004 james}
\end{Highlighting}
\end{Shaded}

If the ports are already taken by other containers, stop and remove all containers:

\begin{Shaded}
\begin{Highlighting}[]
\ExtensionTok{docker}\NormalTok{ stop }\VariableTok{$(}\ExtensionTok{docker}\NormalTok{ ps -a -q}\VariableTok{)}
\ExtensionTok{docker}\NormalTok{ rm }\VariableTok{$(}\ExtensionTok{docker}\NormalTok{ ps -a -q}\VariableTok{)}
\end{Highlighting}
\end{Shaded}

Reissue the \texttt{docker\ run}, and the container should now run. Check by

\begin{Shaded}
\begin{Highlighting}[]
\ExtensionTok{docker}\NormalTok{ ps}
\end{Highlighting}
\end{Shaded}

which should list a container created from the \texttt{james} image.

\hypertarget{checks-with-the-browser}{%
\section{Checks with the browser}\label{checks-with-the-browser}}

\begin{Shaded}
\begin{Highlighting}[]
\ExtensionTok{http}\NormalTok{://localhost}
\end{Highlighting}
\end{Shaded}

should show Apache2 Ubuntu default screen.

\begin{Shaded}
\begin{Highlighting}[]
\ExtensionTok{http}\NormalTok{://localhost/ocpu/test/}
\end{Highlighting}
\end{Shaded}

should show \texttt{OpenCPU} test page.

\begin{Shaded}
\begin{Highlighting}[]
\ExtensionTok{http}\NormalTok{://localhost/rstudio/}
\end{Highlighting}
\end{Shaded}

should start the Rstudio IDE on the container. Use \texttt{opencpu:opencpu} to log in.

\begin{Shaded}
\begin{Highlighting}[]
\ExtensionTok{http}\NormalTok{://localhost/ocpu/library/james/www/}
\end{Highlighting}
\end{Shaded}

should start the JAMES javascript interface.

See also \url{https://registry.hub.docker.com/r/opencpu/rstudio}

\hypertarget{security}{%
\section{Security}\label{security}}

\begin{enumerate}
\def\labelenumi{\arabic{enumi}.}
\tightlist
\item
  The container is shielded from the machine on which it runs. However, the materials within the container are not protected.
\item
  Don't use intermediate containers, since these contain your token in \texttt{/.Renviron}. The latest (\texttt{james}) container does not hold your token, and can be shared, of course, keeping in mind remark 1.
\end{enumerate}

\bibliography{book.bib,packages.bib}

\end{document}
